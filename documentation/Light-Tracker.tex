\documentclass[10pt, openany]{book}

\usepackage{fancyhdr}
\usepackage{imakeidx}

\usepackage{amsmath}
\usepackage{amsfonts}

\usepackage{geometry}
\geometry{letterpaper}

\usepackage{fancyvrb}
\usepackage{fancybox}

\usepackage{gensymb}

\usepackage{url}
%
% Rules to allow import of graphics files in EPS format
%
\usepackage{graphicx}
\DeclareGraphicsExtensions{.eps}
\DeclareGraphicsRule{.eps}{eps}{.eps}{}
%
%  Include the listings package
%
\usepackage{listings}
%
%  Define Tiny Lisp based on Common Lisp
%
\lstdefinelanguage[Tiny]{Lisp}[]{Lisp}{morekeywords=[13]{atomp, bit-vector-p, car, cdr, char-downcase, char-code, char-upcase, compiled-function-p, dowhile, dump, exit, fresh-line, if, code-char, lambda, msg, nullp, parse-integer, peek8, peek16, peek32, poke8, poke16, poke32, quote, read-line, reset, setq, simple-bit-vector-p, simple-string-p, simple-vector-p, string-downcase, string-upcase}}
%
% Macro definitions
%
\newcommand{\operation}[1]{\textbf{\texttt{#1}}}
\newcommand{\package}[1]{\texttt{#1}}
\newcommand{\function}[1]{\texttt{#1}}
\newcommand{\constant}[1]{\emph{\texttt{#1}}}
\newcommand{\keyword}[1]{\texttt{#1}}
\newcommand{\datatype}[1]{\texttt{#1}}
\newcommand{\tl}{Tiny-Lisp}
\newcommand{\cl}{Common Lisp}
%
% Front Matter
%
\title{Light Tracker}
\author{Brent Seidel \\ Phoenix, AZ}
\date{ \today }
%========================================================
%%% BEGIN DOCUMENT
\begin{document}
%
% Produce the front matter
%
\frontmatter
\maketitle
\begin{center}
This document is \copyright 2021 Brent Seidel.  All rights reserved.

\paragraph{}Note that this is a draft version and not the final version for publication.
\end{center}
\tableofcontents

\mainmatter
%----------------------------------------------------------
\chapter{Introduction}
The light tracker was conceived as a way to test a 3-D printed pan and tilt platform.

%----------------------------------------------------------
\chapter{Hardware}
\section{Electronics}
\subsection{Micro-controller Requirements}
There are four photo-transistors light sensors, thus 4 analog inputs are required.  Each stepper motor requires 4 digital output lines.  For the two steppers, a total of 8 outputs are required.  Should limit switches be added, a few digital inputs will also be required.  The micro-controller that I'm using is an Arduino Due, which has ample resources for this project.

\section{Mechanical}
\subsection{Gearing}
Both stepper motors have a 20 tooth gear and are 400 steps per revolution.

The tilt gear has 50 teeth giving a $\frac{2}{5}$ gear ratio.  With the gearing, 1000 steps of the motor will completely rotate the tilt frame.  This is 0.36\degree per step or 2.78 steps per degree.

The pan gear has 100 teeth giving a $\frac{1}{5}$ gear ratio.  With the gearing, 2000 steps of the motor will completely rotate the pan head, giving 0.18\degree per steps or 5.56 steps per degree.

%----------------------------------------------------------
\chapter{Software}

\end{document}

